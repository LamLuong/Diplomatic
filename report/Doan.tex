\documentclass[13pt, a4paper]{extreport}
\usepackage[utf8]{vietnam}  
\usepackage{type1cm}
\usepackage{listings}
\usepackage{fancybox}
\usepackage{fancyhdr}
\usepackage[left=3.40cm, right=2.00cm, top=2cm, bottom=2cm]{geometry}
\usepackage{graphicx}
\linespread{1.5}
\usepackage{mathtools}
\usepackage{mathptmx}
\usepackage{amsmath} %large sum
\usepackage{relsize} %large sum
\usepackage{etoolbox}
\usepackage{titlesec}
\usepackage{indentfirst}
\usepackage{enumitem}
\usepackage{float} %keep fixgure under text
\usepackage{tikz}

\usetikzlibrary{fit,positioning}

\fancypagestyle{plain}{
  \fancyhead{} 
  \renewcommand{\headrulewidth}{0pt}
  \fancyfoot{}
  \fancyfoot[LE,RO]{Trang \thepage} 
  \fancyfoot[RE,LO]{Sinh viên thực hiện: Lương Tiến Lâm – 20121957 – K57 – CNTT-TT2 2.04}
  \renewcommand{\footrulewidth}{0.1pt}
}
\renewcommand{\baselinestretch}{1.0}


\titleformat{\chapter}[block]
  {\normalfont\LARGE\bfseries}{\chaptertitlename\ \thechapter:}{0pt}{\Large}[{\vspace{0ex}\titlerule[2pt]}]

\titleformat{\section}
  {\normalfont\fontsize{14}{16.8}\bfseries}{\thesection}{1em}{}
\titleformat{\subsection}
  {\normalfont\fontsize{14}{16.8}\bfseries}{\thesubsection}{1em}{}

\titlespacing*{\chapter}{0pt}{0pt}{5pt}

\setlist{parsep=0pt,listparindent=\parindent}

\begin{document}
\thispagestyle{empty}
\thisfancypage{
\setlength{\fboxsep}{3pt}
\fbox}{}

\pagestyle{plain}
\begin{center}
{\fontsize{16}{19.2}\selectfont{TRƯỜNG ĐẠI HỌC BÁCH KHOA HÀ NỘI \\ VIỆN CÔNG NGHỆ THÔNG TIN - TRUYỀN THÔNG}} \\
\textbf{------------------------  *  ------------------------}\\
\vspace{1.3in}
\begin{center}
{\fontsize{32pt}{38.4}\selectfont {ĐỒ ÁN}}\\
\vspace{0.05in}
{\fontsize{38pt}{45.6}\selectfont \textbf{TỐT NGHIỆP ĐẠI HỌC}}\\
\vspace{0.12in}
{\fontfamily{phv} \fontsize{20pt}{24}\selectfont {NGÀNH CÔNG NGHỆ THÔNG TIN}}\\
\vspace{1.3in}
{\fontsize{22pt}{26.4}\selectfont {\textbf{XÂY DỰNG MÔ HÌNH HPC TRONG ỨNG DỤNG GÁN ĐA NHÃN ẢNH}}}
\end{center}


\vspace{0.45in}
\begin{flushleft}
{\hspace{1.8in}{\fontsize{14pt}{16.8}\selectfont {Sinh viên thực hiện:\hspace{0.12in} \textbf{Lương Tiến Lâm}}}}\\
{\hspace{1.8in}\hspace{1.215in}\fontsize{14pt}{16.8}\selectfont {Lớp: CNTT-TT 2.04 - K57}}\\
{\hspace{1.8in}{\fontsize{14pt}{16.8}\selectfont {Giáo viên hướng dẫn: TS \textbf{Nguyễn Thị Oanh}}}}\\
\end{flushleft}
\end{center}
\vspace{1.9in}
\begin{center}
{\fontsize{16pt}{1}\selectfont Hà Nội 5-2017 }\\
\end{center}

\newpage
\begin{center}
{\fontsize{16pt}{19.2}\selectfont \textbf{PHIẾU GIAO NHIỆM VỤ ĐỒ ÁN TỐT NGHIỆP}}
\end{center}	
\begin{enumerate}
  \fontsize{13pt}{15.6}\selectfont
  \item Thông tin về sinh viên \\
   Họ và tên sinh viên: LƯƠNG TIẾN LÂM \\
   Điện thoại liên lạc: 01245021194 \hspace{2cm} Email: lamluongbka@gmail.com \\
   Lớp:CNTT-TT2 2.04 \hspace{4.28cm} Hệ đào tạo: Đại học chính quy \\
   Đồ án tốt nghiệp được thực hiện tại: Bộ môn Khoa học máy tính, Viện Công nghệ thông tin và truyền thông, Đại học Bách Khoa Hà Nội.
  Thời gian làm ĐATN: Từ ngày 12/01/2017 đến 29/05/2017
  \item Mục đích nội dung của ĐATN \\
  \begin{itemize}
    \item Nghiên cứu mạng học sâu GGNet VGG
    \item Nghiên cứ thuật toán BING
    \item Áp dụng mạng học sâu trong bài toán gán đa nhãn ảnh
  \end{itemize}
  \item Các nhiệm vụ cụ thể của ĐATN \\
  \item Lời cam đoan của sinh viên:\\
   Tôi – Lương Tiến Lâm - cam kết ĐATN là công trình nghiên cứu của bản thân tôi dưới sự hướng dẫn của TS.Nguyễn Thị Oanh. \\
   Các kết quả nêu trong ĐATN là trung thực, không phải là sao chép toàn văn của bất kỳ công trình nào khác.
  \begin{flushleft}
    {\fontsize{13pt}{15.6pt}\selectfont {
      \hspace{3in}Hà Nội, ngày \hspace{0.5cm} tháng \hspace{0.5cm} năm 2017 \\
      \hspace{3.7in}Tác giả ĐATN \\
      \hspace{3.5in}\textit{(Ký và ghi rõ họ tên)}\\[1.5cm]}
    }
  \end{flushleft}
  \item Xác nhận của giáo viên hướng dẫn về mức độ hoàn thành của ĐATN và cho phép bảo vệ:
    \begin{flushleft}
	  {\fontsize{13pt}{15.6pt}\selectfont {
        \hspace{3in}Hà Nội, ngày \hspace{0.5cm} tháng \hspace{0.5cm} năm 2017 \\
        \hspace{3.5in}Giáo viên hướng dẫn \\
        \hspace{3.5in}\textit{(Ký và ghi rõ họ tên)}\\[1.5cm]
        }
      }
  	\end{flushleft}
\end{enumerate}

\newpage
\chapter*{\centerline{TÓM TẮT NỘI DUNG ĐỒ ÁN TỐT NGHIỆP}}
\fontsize{13}{15.6}\selectfont
\setlength{\parindent}{0.7cm}
Thời đại bùng nổ về công nghệ thông tin con người đang không những chỉ muốn các công cụ mình sử dụng phải thật thuận tiên, giảm thời gian thao tác mà nó còn phải thông minh, có thể biết được người dung muôns gì ngay cả khi chưa được ra lệnh. Quả thực trong thời gian qua, ngành Trí Tuệ Nhân Tạo nói chung và Học Máy nói riêng đang có bước phát triển đột phá không ngừng, nhờ sự phát triển đó mà các thiết bị mà con nguwoif sử dụng đang ngày càng tỏ ra thông minh vượt bậc thậm chí là vượt qua cả sự nhận thức của con người.\\
\indent Mạng học sâu là một nhánh của ngành khoa học máy tính kể trên và bài toán huấn luyện cho máy tính về một tri thức dựa trên một tập dữ liệu sẵn có không phải bài toán mới nhưng nó vẫn đang phát triển không ngừng và tương lai còn hứa hẹn ra những thành tựu lớn.\\
\indent Trong đồ án này tác giả tập trung vào bài toán gán đa nhãn ảnh mà cụ thể là gán nhãn cho bức ảnh chụp nhiều đồ ăn, đưa ra nhiều và đúng nhất có thể các món ăn có trong bức ảnh dựa vào Mạng học sâu và thuật toán nhận dạng các đối tượng trong ảnh.\\
\indent Nôi dung đồ án gồm các chương chính sau:

\newpage
\chapter*{\centerline{LỜI NÓI ĐẦU}}
\indent Nhận dạng nội dung hay khái quát hơn là gán nhãn cho ảnh là bài toán đã được quan tâm từ lâu và đã vượt qua phạm trù lý thuyết đơn thuần và tiến tới ứng dụng thực tế. có rất nhiều ứng dụng dựa trên detech và gán nhãn ảnh được các ông lớn sử dụng và rất thành công dù chỉ là opensorce nhưng cũng kiếm được các khoản lợi nhuận khổng lồ như tag ảnh của fb hay google draw của google\\
Nói về ẩm thực thì chúng ta đã biết độ phong phú và đa dạng của lĩnh vực này, có hàng ngàn các món ăn trên thế giớ trải dài từ bán cầu Tây sang Đông và ngay cả ở Việt Nam hệ thống ẩm thực khá phong phú nhwung không phải món ăn nào ta cũng biết. \\
\indent Việc gán nhãn cho các món ăn hiên đang rất được các nhà phát triển web. app quan tâm. và mạng xã hội ẩm thực hiện nay mới chỉ dừng ở mức độ   gán nhãn ảnh dựa trên nhãn mà người dùng gãn sẵn vào. việc up một bức ảnh lên mà người dùng không gán nhãn thì vẫn đang chưa thực hiện được. \\
\indent Ở trong bài toán này yêu cầu được đưa ra như sau: đưa một ảnh đầu vào chụp một hoặc nhiều món ăn. nhiệm vụ của máy tính phải phát hiện được nhiều món ăn nhất có thể và gán nhãn cho ảnh là tên các mon ăn mà nó phát hiên được.
Có rất nhiều mô hình, mạng học sâu có thê giải quyết được bài toán này nhưng trong khuôn khổ đồ án tác gỉa sử dụng mô hình mạng googlenet để làm mô hình tranning và detecting, thuật tón BING để khoanh vùng các món ăn trong ảnh.

\tableofcontents
\newpage

\newpage
\chapter{\centerline{ĐẶT VẤN ĐỀ}}
\section{Bối cảnh}
\indent Nhu cầu ăn uống của con người từ xa xưa đã là một nét văn hóa, con người luôn muốn thông tin cho nhau những món ăn ngon, những địa điểm hấp dẫn. Mạng xã hội ẩm thực ra đời để đáp ứng nhu cầu đó, với sự ra đời ngày càng nhiều của các trang mạng chia sẻ đồ ăn (Lozi, foody ...) người ta đã có thể dễ dàng chia sẻ từng món ăn, địa chỉ và các khoảnh khắc checkin cho người thân, bạn bè hoặc "pr" cho quán ăn của mình.\\
\indent Cũng xuất phát từ nhu cầu người dùng, các trang mạng xã hội cũng ngày càng cải thiện về tính năng, giao diện và khả năng tương tác với người sử dụng. Từ việc up ảnh rồi "tag" bạn bè hay việc đánh dấu địa chỉ chụp bức ảnh tất cả đều đc các nhà phát triển cài đặt làm cho thế giới mạng xã hội ẩm thực chưa bao giờ hết sôi động.\\
\indent Tuy nhiên ở các trang mạng xã hội ẩm thực hiện nay mới chỉ dừng lại ở mức người dùng tự đăng ảnh rồi tự gán tên món ăn cho bức ảnh, đối với những người "lười", "ngại gõ phím" hoặc tên món ăn mà người dùng không biết hoặc không nhớ tên mà không có tờ thực đơn trong khi muốn đăng ảnh ngay trước khi ăn thì vấn đề tuy nhỏ này cũng phải khiến các nhà phát triển để tâm và nghiên cứu thêm về tính năng hỗ trợ công việc này.\\
\indent Như vậy bài toán từ nhu cầu thực tế trên, bài toán mà ta phải giúp máy tính giải quyết là: Đưa ra gợi ý tên món ăn mà người dùng đưa lên. Tuy nhiên, trong một bàn ăn sẽ không thể chỉ có một món ăn, như thế bài toán được mở rộng hơn là gợi ý nhiều nhất có thể các món có trong bức ảnh mà người dùng đưa lên.
\section{Bài toán}
\indent Bài toán mà em sẽ thực trong đồ án này là:\\
\indent \textit{Từ một ảnh chụp các món ăn đầu vào, gán cho ảnh các nhãn là tên món ăn có trong ảnh.}
\section{Hướng giải quyết}
\indent Thể thực hiện bài toán trên ta cần thực hiện hai bài toán nhỏ hơn:
\begin{itemize}
	\item Detect objects: Tìm kiếm các đối tượng đồ ăn có trong ảnh, đánh dấu vị trí vẽ đường biên sao cho đường biên này bao quát vừa đủ món ăn (trong đường biên đó chỉ có một đối tượng món ăn). (1)
	\item Thực hiện gán nhãn cho từng đối tượng đã detect được ở (1), nhãn này chính là tên của món ăn. Nhãn của ảnh đầu vào chính là các nhãn của các đối tượng thành phần. (2)
	"ảnh minh họa"
\end{itemize}
\indent Để thực hiện được (1) chúng ta cần một thuật toán để detect objects, thông thường các điểm ảnh biểu diến một đối tượng trong ảnh sẽ có một số điểm chung nhất định. Và thuật toán sử dụng cần phải tìm ra được các điểm chung đấy. Một số phương pháp có thể áp dụng như BING (Binarized Normed Gradients), deconvolution neural network...\\
\indent Trong (2) rõ ràng đây là một bài toán phân loại dữ liệu có thể thực hiện bằng phương pháp học máy học có giám sát: Sử dụng một mạng học sâu để học các tham số phân loại các input đầu vào dựa trên một tập học cho trước.
\chapter{\centerline{KIẾN THỨC NỀN TẢNG}}
\section{Mạng NơRon}
\section{Mạng nơron nhân chập}

\end{document}















